\documentclass[uplatex, a4j, 11pt, fleqn, dvipdfmx]{article}

\usepackage{amsmath}
\usepackage{amssymb}
\usepackage[top=3cm, bottom=3cm, left=1cm, right=1cm]{geometry}

% See: https://tex.stackexchange.com/questions/103885/how-to-type-an-inline-chi-in-latex
\newcommand{\irchi}[2]{\raisebox{\depth}{$#1\chi$}}
\DeclareRobustCommand{\rchi}{{\mathpalette\irchi\relax}}

\begin{document}

\begin{center}
	{\Huge システム計画論 第 8 回 課題}
\end{center}

\begin{flushright}
	{\Large \today ~~~~ 29C23002 ~~~ 石川健太郎}
\end{flushright}

\begin{align}
	\begin{aligned}
		\text{maximize} ~~   &
		\min \left(
		\mathrm{Nes} \left( c_1 x_1 + c_2 x_2 \geq 450 \right), ~~
		\mathrm{Pos} \left( c_1 x_1 + c_2 x_2 \geq 530 \right)
		\right),
		\\
		\text{subject to} ~~ &
		\mathrm{Nes} \left( a_1 x_1 + b_1 x_2 \leq 240 \right) \geq 0.8,
		\\
		                     & \mathrm{Nes} \left( a_2 x_1 + b_2 x_2 \leq 400 \right) \geq 0.8,
		\\
		                     & \mathrm{Nes} \left( a_3 x_1 + b_3 x_2 \leq 210 \right) \geq 0.8,
		\\
		                     & x_1, x_2 \geq 0.
	\end{aligned}
	\label{fom-1}
\end{align}

\begin{align}
	\tilde{A}_1 & = \langle 2, 0.7 \rangle
	\label{fom-2}                          \\
	\tilde{A}_2 & = \langle 4, 1.5 \rangle
	\label{fom-3}                          \\
	\tilde{A}_3 & = \langle 1, 0.5 \rangle
	\label{fom-4}                          \\
	\tilde{B}_1 & = \langle 3, 0.5 \rangle
	\label{fom-5}                          \\
	\tilde{B}_2 & = \langle 2, 0.3 \rangle
	\label{fom-6}                          \\
	\tilde{B}_3 & = \langle 3, 0.3 \rangle
	\label{fom-7}                          \\
	\tilde{B}_1 & = \langle 5, 1 \rangle
	\label{fom-8}                          \\
	\tilde{B}_1 & = \langle 7, 0.7 \rangle
	\label{fom-9}                          \\
\end{align}



\end{document}
