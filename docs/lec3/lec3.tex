\documentclass[uplatex, a4j, 10pt, fleqn, dvipdfmx]{article}

\usepackage{amsmath}
\usepackage{amssymb}
\usepackage[top=3cm, bottom=3cm, left=1cm, right=1cm]{geometry}

% See: https://tex.stackexchange.com/questions/103885/how-to-type-an-inline-chi-in-latex
\newcommand{\irchi}[2]{\raisebox{\depth}{$#1\chi$}}
\DeclareRobustCommand{\rchi}{{\mathpalette\irchi\relax}}

\begin{document}

\begin{center}
	{\Huge システム計画論 第 3 回 課題}
\end{center}

\begin{flushright}
	{\Large \today ~~~~ 29C23002 ~~~ 石川健太郎}
\end{flushright}

\section*{[ 3a ]}

\noindent
$N = \{ 1, 2, \dots, n \}, ~~ N^- = \{ i \in N ~ | ~ k_i < 0 \}, ~~ N^+ = \{ i \in N ~ | ~ k_i \geq 0 \}$ とする.
\begin{align}
	\sum_{i \in N} k_i A_i
	 & = \sum_{i \in N^-} k_i A_i + \sum_{i \in N^+} k_i A_i
	\\
	 & = - \sum_{i \in N^-} \left| k_i \right| A_i + \sum_{i \in N^+} \left| k_i \right| A_i
	\\
	 & = - \sum_{i \in N^-} \left[ \left| k_i \right| a_i^\text{L}, ~ \left| k_i \right| a_i^\text{R} \right] +
	\sum_{i \in N^+} \left[ \left| k_i \right| a_i^\text{L}, ~ \left| k_i \right| a_i^\text{R} \right]
	\\
	 & = - \left[
		\sum_{i \in N^-} \left| k_i \right| a_i^\text{L}, ~ \sum_{i \in N^-} \left| k_i \right| a_i^\text{R}
		\right] + \left[
		\sum_{i \in N^+} \left| k_i \right| a_i^\text{L}, ~ \sum_{i \in N^+} \left| k_i \right| a_i^\text{R}
		\right]
	\\
	 & = \left[
		- \sum_{i \in N^-} \left| k_i \right| a_i^\text{R}, ~ - \sum_{i \in N^-} \left| k_i \right| a_i^\text{L}
		\right] + \left[
		\sum_{i \in N^+} \left| k_i \right| a_i^\text{L}, ~ \sum_{i \in N^+} \left| k_i \right| a_i^\text{R}
		\right]
	\\
	 & = \left[
		\sum_{i \in N^+} \left| k_i \right| a_i^\text{L} - \sum_{i \in N^-} \left| k_i \right| a_i^\text{R}, ~
		\sum_{i \in N^+} \left| k_i \right| a_i^\text{R} - \sum_{i \in N^-} \left| k_i \right| a_i^\text{L}, ~
		\right]
\end{align}

% https://slideplayer.com/slide/5937402/
\section*{[ 3b ]}

\noindent
まず, $\left( f \left( \tilde{A}_1, \tilde{A}_2 \right) \right)_h \supseteq
	f \left( ( \tilde{A}_1 )_h, ( \tilde{A}_2 )_h \right)$ を示す.

\noindent
$y \in f \left( ( \tilde{A}_1 )_h, ( \tilde{A}_2 )_h \right)$ とする.

\noindent
$y$ の定義から任意の $y$ に対して $y = f \left( \bar{x}_1, \bar{x}_2 \right)$ を満たす
$\bar{x}_1 \in ( \tilde{A} )_h, ~ \bar{x}_2 \in ( \tilde{A} )_h$ が存在するので,
任意の $y$ について $f^{-1}(y) \neq \emptyset$ である.

\noindent
よって,拡張原理から次の関係が成り立つ.
\begin{align}
	\mu_{f \left( \tilde{A}_1, \tilde{A}_2 \right)}(y)
	 & = \sup_{\left( x_1, x_2 \right) \in f^{-1}(y)} \min \left( \mu_{\tilde{A}_1}(x_1), \mu_{\tilde{A}_2}(x_2) \right)
	\\
	 & \geq \min \left( \mu_{\tilde{A}_1}(\bar{x}_1), \mu_{\tilde{A}_2}(\bar{x}_2) \right)
	\\
	 & > h ~~~~
	\left( \bar{x}_1 \in ( \tilde{A} )_h, ~ \bar{x}_2 \in ( \tilde{A} )_h ~ \text{なので,} ~ \mu_{\tilde{A}_1}(\bar{x}_1) > h, ~ \mu_{\tilde{A}_2}(\bar{x}_2) > h \right)
\end{align}

\noindent
$\mu_{f \left( \tilde{A}_1, \tilde{A}_2 \right)}(y) \geq h$ なので, $y \in \left( f \left( \tilde{A}_1, \tilde{A}_2 \right) \right)_h$ である.

\noindent
以上から, $\left( f \left( \tilde{A}_1, \tilde{A}_2 \right) \right)_h \supseteq
	f \left( ( \tilde{A}_1 )_h, ( \tilde{A}_2 )_h \right)$ が示された.

\noindent
次に, $\left( f \left( \tilde{A}_1, \tilde{A}_2 \right) \right)_h \subseteq
	f \left( ( \tilde{A}_1 )_h, ( \tilde{A}_2 )_h \right)$ を示す.

\noindent
$y \in \left( f \left( \tilde{A}_1, \tilde{A}_2 \right) \right)_h$ とする.

\noindent
拡張原理から, $\mu_{f\left( \tilde{A}_1, \tilde{A}_2 \right)}(y) = \min \left( \mu_{\tilde{A}_1}(\bar{x}_1), \mu_{\tilde{A}_2}(\bar{x}_2) \right)$
を満たす $\bar{x}_1 \in \tilde{A}_1, \bar{x}_2 \in \tilde{A}_2$ が存在する.


\noindent
$y \in \left( f \left( \tilde{A}_1, \tilde{A}_2 \right) \right)_h$ より, $\mu_{f\left( \tilde{A}_1, \tilde{A}_2 \right)}(y) > h$ なので,
$\min \left( \mu_{\tilde{A}_1}(\bar{x}_1), \mu_{\tilde{A}_2}(\bar{x}_2) \right) > h$ である.

\noindent
すなわち, $\mu_{\tilde{A}_1}(\bar{x}_1) > h, \mu_{\tilde{A}_2}(\bar{x}_2) > h$ が成り立ち,
$\bar{x}_1 \in ( \tilde{A}_1 )_h, \bar{x}_2 \in ( \tilde{A}_2 )_h$ である.

\noindent
このことから, $y \in f \left( ( \tilde{A}_1 )_h, ( \tilde{A}_2 )_h \right)$ であるといえる.

\noindent
以上から, $\left( f \left( \tilde{A}_1, \tilde{A}_2 \right) \right)_h \subseteq
	f \left( ( \tilde{A}_1 )_h, ( \tilde{A}_2 )_h \right)$ が示された.

\noindent
$\left( f \left( \tilde{A}_1, \tilde{A}_2 \right) \right)_h \supseteq
	f \left( ( \tilde{A}_1 )_h, ( \tilde{A}_2 )_h \right), \left( f \left( \tilde{A}_1, \tilde{A}_2 \right) \right)_h \subseteq
	f \left( ( \tilde{A}_1 )_h, ( \tilde{A}_2 )_h \right)$ が成り立つので, $\left( f \left( \tilde{A}_1, \tilde{A}_2 \right) \right)_h =
	f \left( ( \tilde{A}_1 )_h, ( \tilde{A}_2 )_h \right)$ が成り立つ.

% \noindent
% $y \in \left( f \left( \tilde{A}_1, \tilde{A}_2 \right) \right)_h$ とする.
% \begin{align}
% 	                & \mu_{f \left( \tilde{A}_1, \tilde{A}_2 \right)}(y) > h
% 	\\
% 	\Leftrightarrow & \sup_{\left( x_1, x_2 \right) \in f^{-1}(y)}
% 	\min \left( \mu_{\tilde{A}_1}(x_1), \mu_{\tilde{A}_2}(x_2) \right) > h ~~~~
% 	\left( y \text{の定義から任意の} y \text{について} f^{-1}(y) \neq \emptyset \text{であることと拡張原理より} \right)
% 	\\
% 	\Rightarrow     & \min \left( \mu_{\tilde{A}_1}(\bar{x}_1), \mu_{\tilde{A}_2}(\bar{x}_2) \right) > h ~~~~
% 	\left( \min \left( \mu_{\tilde{A}_1}(\bar{x}_1), \mu_{\tilde{A}_2}(\bar{x}_2) \right) > \min \left( \mu_{\tilde{A}_1}(x_1), \mu_{\tilde{A}_2}(x_2) \right), ~ \forall \left( x_1, x_2 \right) \in f^{-1}(y), ~ x_1 \in \tilde{A}_1, x_2 \in \tilde{A}_2 \right)
% 	\\
% 	\Rightarrow     & \mu_{\tilde{A}_1}(\bar{x}_1) > h ~ \text{かつ} ~ \mu_{\tilde{A}_2}(\bar{x}_2) > h
% \end{align}

\end{document}
