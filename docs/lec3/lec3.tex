\documentclass[uplatex, a4j, 10pt, fleqn, dvipdfmx]{article}

\usepackage{amsmath}
\usepackage{amssymb}
\usepackage[top=3cm, bottom=3cm, left=1cm, right=1cm]{geometry}

% See: https://tex.stackexchange.com/questions/103885/how-to-type-an-inline-chi-in-latex
\newcommand{\irchi}[2]{\raisebox{\depth}{$#1\chi$}}
\DeclareRobustCommand{\rchi}{{\mathpalette\irchi\relax}}

\begin{document}

\begin{center}
	{\Huge システム計画論 第 3 回 課題}
\end{center}

\begin{flushright}
	{\Large \today ~~~~ 29C23002 ~~~ 石川健太郎}
\end{flushright}

\section*{[ 3a ]}

\noindent
$N = \{ 1, 2, \dots, n \}, ~~ N^- = \{ i \in N ~ | ~ k_i < 0 \}, ~~ N^+ = \{ i \in N ~ | ~ k_i \geq 0 \}$ とする.
\begin{align}
	\sum_{i \in N} k_i A_i
	 & = \sum_{i \in N^-} k_i A_i + \sum_{i \in N^+} k_i A_i
	\\
	 & = - \sum_{i \in N^-} \left| k_i \right| A_i + \sum_{i \in N^+} \left| k_i \right| A_i
	\\
	 & = - \sum_{i \in N^-} \left[ \left| k_i \right| a_i^\text{L}, ~ \left| k_i \right| a_i^\text{R} \right] +
	\sum_{i \in N^+} \left[ \left| k_i \right| a_i^\text{L}, ~ \left| k_i \right| a_i^\text{R} \right]
	\\
	 & = - \left[
		\sum_{i \in N^-} \left| k_i \right| a_i^\text{L}, ~ \sum_{i \in N^-} \left| k_i \right| a_i^\text{R}
		\right] + \left[
		\sum_{i \in N^+} \left| k_i \right| a_i^\text{L}, ~ \sum_{i \in N^+} \left| k_i \right| a_i^\text{R}
		\right]
	\\
	 & = \left[
		- \sum_{i \in N^-} \left| k_i \right| a_i^\text{R}, ~ - \sum_{i \in N^-} \left| k_i \right| a_i^\text{L}
		\right] + \left[
		\sum_{i \in N^+} \left| k_i \right| a_i^\text{L}, ~ \sum_{i \in N^+} \left| k_i \right| a_i^\text{R}
		\right]
	\\
	 & = \left[
		\sum_{i \in N^+} \left| k_i \right| a_i^\text{L} - \sum_{i \in N^-} \left| k_i \right| a_i^\text{R}, ~
		\sum_{i \in N^+} \left| k_i \right| a_i^\text{R} - \sum_{i \in N^-} \left| k_i \right| a_i^\text{L}, ~
		\right]
\end{align}

% https://slideplayer.com/slide/5937402/
\section*{[ 3b ]}

\noindent
$y \in \left( f \left( \tilde{A}_1, \tilde{A}_2 \right) \right)_h$ のとき,
$y$ の定義から任意の $y$ に対して $y = f \left( x_1, x_2 \right)$ を満たす
$x_1 \in \tilde{A}, ~ x_2 \in \tilde{A}$ が存在するので,
任意の $y$ について $f^{-1}(y) \neq \emptyset$ である.
\begin{equation}
	y \in \left( f \left( \tilde{A}_1, \tilde{A}_2 \right) \right)_h
	\Rightarrow
	f^{-1}(y) \neq \emptyset
	\label{fom2-1}
\end{equation}

\begin{align}
	                & ~ y \in f \left( ( \tilde{A}_1 )_h, ( \tilde{A}_2 )_h \right)
	\\
	\Leftrightarrow & ~ \exists \left( \bar{x}_1, \bar{x_2} \right) \in f^{-1}(y); ~~ \bar{x}_1 \in ( \tilde{A}_1 )_h ~~ \text{amd} ~~ \bar{x}_2 \in ( \tilde{A}_2 )_h
	\\
	\Leftrightarrow & ~ \exists \left( \bar{x}_1, \bar{x_2} \right) \in f^{-1}(y); ~~ \mu_{\tilde{A}_1}(\bar{x}_1) > h ~~ \text{amd} ~~ \mu_{\tilde{A}_2}(\bar{x}_2) > h
	\\
	\Leftrightarrow & ~ \exists \left( \bar{x}_1, \bar{x_2} \right) \in f^{-1}(y); ~~ \min \left( \mu_{\tilde{A}_1}(\bar{x}_1), \mu_{\tilde{A}_2}(\bar{x}_2) \right) > h
	\label{fom2-2}
\end{align}

\begin{align}
	                & ~ y \in \left( f \left( \tilde{A}_1, \tilde{A}_2 \right) \right)_h
	\\
	\Leftrightarrow & ~ \mu_{f\left( \tilde{A}_1, \tilde{A}_2 \right)}(y) > h
	\\
	\Leftrightarrow & ~ \sup_{\left( x_1, x_2 \right) \in f^{-1}(y)} \min \left( \mu_{\tilde{A}_1}(x_1), \mu_{\tilde{A}_2}(x_2) \right) > h
	\qquad \left( \because (\ref{fom2-1}) \text{と拡張原理} \right)
	\label{fom2-3}
\end{align}

\noindent
任意の $\left( \bar{x}_1, \bar{x_2} \right) \in f^{-1}(y)$ は $\sup_{\left( x_1, x_2 \right) \in f^{-1}(y)} \min \left( \mu_{\tilde{A}_1}(x_1), \mu_{\tilde{A}_2}(x_2) \right)$ よりも小さいので, (\ref{fom2-2}) $~\Leftarrow~$ (\ref{fom2-3}) が成り立つ.

\noindent
$\sup$ の定義から,明らかに(\ref{fom2-2}) $~\Rightarrow~$ (\ref{fom2-3}) が成り立つ.

\noindent
よって,(\ref{fom2-2}) $~\Leftrightarrow~$ (\ref{fom2-3}) が成り立つ.
以上から, $y \in f \left( ( \tilde{A}_1 )_h, ( \tilde{A}_2 )_h \right) \Leftrightarrow y \in \left( f \left( \tilde{A}_1, \tilde{A}_2 \right) \right)_h$ が示された.

\end{document}
