\documentclass[uplatex, a4j, 10pt, fleqn, dvipdfmx]{article}

\usepackage{amsmath}
\usepackage{amssymb}
\usepackage[top=3cm, bottom=3cm, left=1cm, right=1cm]{geometry}

% See: https://tex.stackexchange.com/questions/103885/how-to-type-an-inline-chi-in-latex
\newcommand{\irchi}[2]{\raisebox{\depth}{$#1\chi$}}
\DeclareRobustCommand{\rchi}{{\mathpalette\irchi\relax}}

\begin{document}

\begin{center}
	{\Huge システム計画論 第 5 回 課題}
\end{center}

\begin{flushright}
	{\Large \today ~~~~ 29C23002 ~~~ 石川健太郎}
\end{flushright}

\section*{[ 5a ]}

\begin{equation}
	N_{\tilde{A}}(\tilde{B}) \geq \Pi_{\tilde{A}}(\tilde{B})
\end{equation}

\begin{equation}
	\tilde{B} ~ \text{がクリスプ集合であるとき,} ~~ N_{\tilde{A}}(\tilde{B}) \geq h ~ \Leftrightarrow ~ (\tilde{A})_{1-h} \subseteq [\tilde{B}]_h
\end{equation}



\end{document}
